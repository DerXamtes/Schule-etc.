\documentclass[a4paper]{article}

\usepackage[utf8]{inputenc}
\usepackage[T1]{fontenc}
\usepackage[ngerman]{babel}

\usepackage{amsmath}
\usepackage[left=3cm, right=3cm, top=3cm, bottom=2.5cm]{geometry}

\usepackage{tikz}
\usepackage{pgfplots}

\usepackage{hyperref}
\hypersetup{
    colorlinks=true,
    linkcolor=black,
    filecolor=magenta,
    urlcolor=blue,
    citecolor=green,
    pdftitle={Maximilian Ansorg Korrektur Mathearbeit}
}

\usepackage[headsepline]{scrlayer-scrpage}
\ihead{\normalfont\sffamily Maximilian Ansorg}
\chead{\normalfont\sffamily Korrektur Mathearbeit}
\ohead{\normalfont\sffamily\today}
\cfoot{\normalfont\sffamily\thepage}

\begin{document}
\sffamily
\pagenumbering{arabic}
\setcounter{page}{1}

\begin{center}
	\section*{\underline{\textbf {2. Klausur Mathematik Einführungsphase}}}
\end{center}
\vspace*{5mm}

\subsection*{\textbf{Aufgabe 1: Lineare Funktionen}}
	\begin{minipage}{0.5\textwidth}
		\subsubsection*{a)}
			\begin{align*}
				f(g)&=-\frac{1}{6}x+3\frac{1}{6} \\
				f(f)&=\frac{1}{3}x-2
			\end{align*}
		\subsubsection*{b)}
			\begin{align*}
				h(x)=-2x+4
			\end{align*}
	\end {minipage}
	\begin {minipage}{0.5\textwidth}
		\subsubsection*{c)}
			\begin{align*}
				h(x) &= -2x + 4 \\
				i(x) &= \frac{1}{2}x + b\\
				2	 &= \frac{1}{2}	* (-4) + b\\
				2	 &= -2 + b \\
				4	 &= b\\
				i(x) &= \frac{1}{2}x + 4
			\end{align*}
	\end{minipage}
\vspace*{5mm}

\subsection*{\textbf{Aufgabe 2: Zusammengesetzte Aufgabe lineare Funktionen}}
	\begin{minipage}{1\textwidth}
		\subsubsection*{a)}
			Die Gerade $f(x)$ und $h(x)$ sind orthogonal zueinander und schneiden sich in einem Punkt, \\
			weil $\frac{1}{3} * (-3) = -1$. \\
			Alle Geraden schneiden sich jeweils in einem Punkt, weil sie eine unterschiedliche Steigungen haben.
	\end{minipage}
	\begin{minipage}{0.5\textwidth}
		\vspace*{-5mm}
		\subsubsection*{b)}
			\begin{align*}
				f(x) &= g(x) \\
				\frac{1}{3}x + 1 &= -x \\
				\frac{4}{3}x + 1 &= 0 \\
				\frac{4}{3}x &= -1 \\
				4x &= -3 \\
				x &= -0,75 
				S(-0,75 | 0.75) \\
				m &= \tan(\alpha) \\
				\alpha &= \tan^{-1}(m) \\
				\tan^{-1}(\frac{1}{3}) &= 18.43^{\circ} \\
				\tan^{-1}(-1) &= -45^{\circ}\\
				63,43^{\circ} &= 18.43^{\circ} -(-45^{\circ}) \\ 
			\end{align*}
		\end{minipage}
		\begin{minipage}{0.5\textwidth}
			\vspace*{10mm}
			\subsubsection*{c)}
				\begin{align*}
					\text{$P (3|y)$} \\
					f(x) &= \frac{1}{3}x + 1 \\
					f(3) &= \frac{1}{3} * 3 + 1 \\
					f(3) &= 2 \\
					\text{$P (3|2)$}
				\end{align*}
			\subsubsection*{d)}
			\begin{align*}
				m &= \tan(\alpha) \\
				\tan(20^{\circ}) &= 0,364\\
				k(x) &= 0,364x \\ 
			\end{align*}
			\subsubsection*{e)}
				\[k(3)= 1,091 \] \\
				\hspace*{10mm}
				\text{Punkt liegt nicht auf der Geraden}
		\end{minipage}
	\clearpage

\subsection*{\textbf{Aufgabe 3: Quadratische Funktionen}}
	\begin{minipage}{0.5\textwidth}
		\subsubsection*{a)}
		\begin{align*}	
			\text{Scheitelpunktgform: }f(x) &= (x - 2)^{2} - 1 \\	
			\text{Linearfaktorform: } f(x) &= 1(x - 1)(x - 3)	\\ 
		\end{align*}
		\subsubsection*{b)}
		\begin{align*}
			2 &= a * (-1)^{2} + b * (-1) + c \\
			5 &= a * (-1)^{2} + b * (2)  + c \\
			-1 &= a * (0,5)^{2} + b * (0,5) + c\\ \\
		\end{align*}
		\begin{align*}
			x &= 2 \\
			y &= -1 \\ 
			z &= -1 \\
			g(x) &= 2x^{2} - 1x -1\\
		\end{align*}
		\subsubsection*{c)}
		\begin{align*}
			g(x) &= 2x^{2} - 1x -1\\
			xs &= \frac{b}{2 * a} \\
			xs &= \frac{1}{2 * 2} \\
			xs &= \frac{1}{4}
		\end{align*}
		\begin{align*}
			ys &=  2 * (\frac{1}{4})^{2} - 1(\frac{1}{4}) - 1 \\
			ys &= -\frac{9}{8} \\ \\
			g(x) &= 2 (x - \frac{1}{4}) - \frac{9}{8} \\
		\end{align*}
		\subsubsection*{d)}
		\begin{align*}
			h(x) &= -2(x - 1)(x + 2)\\
			x_{1}&= -1 \\
			x_{2}&= 2 \\
		\end{align*}
	\end{minipage}
	\begin{minipage}{0\textwidth}
		\subsubsection*{e)}
		\begin{align*}
			x_{1/2} &= -\frac{p}{2} \pm\sqrt{(\frac{p}{2})^{2} - q} \\
			x_{1/2} &= \frac{0,5}{2} \pm\sqrt{(\frac{0,5}{2})^{2} + 0,5} \\
			x_{1/2} &= \frac{1}{4} \pm \frac{3}{4} \\
			x_{1} &= \frac{1}{4} + \frac{3}{4} = 1 \\
			x_{2} &= \frac{1}{4} - \frac{3}{4} = -\frac{1}{2} \\
		\end{align*}
		\subsubsection*{f)}
		\begin{align*}
			g(x) &= 2x^{2} - 1x - 1 \\
			t(x) &= 2x - 1 \\
			t(x)&= g(x) \\
			2x - 1&=2x^{2} - 1x - 1 \\
			0 &= 2x^{2} - 3x \\
			\text{TR: } \\
			x_{1} &= \frac{3}{2} \\
			x_{2} &= 0 
		\end{align*}
		\subsubsection*{g)}
		\begin{align*}
			f(x) &= 2x^{2} + 6x - 1 \\
			g(x) &= 2x^{2} \\
		\end{align*}
			\text{Verschiebung y- Achse = +5,5} \\
			\text{Verschiebung x- Achse = +1,5} \\
	\end{minipage}
\clearpage
\newpage

\subsection*{\textbf{Aufgabe 4: Klippenspringen}}
	\begin{figure}[h]
		\begin{minipage}{0\textwidth}
			\subsubsection*{a)}
				\begin{tikzpicture}
					\begin{axis}[
						xmin=0, xmax=8,
						ymin=0, ymax=30,
						xtick={0,1,...,8},
						ytick={0,5,...,30},
						xlabel=$x$,
						ylabel=$y$,
						grid=both,
						grid style={line width=.1pt, draw=gray!10},
						major grid style={line width=.2pt,draw=gray!50},
						]
						
						\addplot[
						domain=0:8,
						samples=100,
						color=red,
						] {-x^2 + 2*x + 27};
						
						\draw[blue, thick] (1,28) -- (1,0);
						\draw[blue, thick] (-1,0) -- (-1,1);
						\draw[blue, thick] (5,0) -- (5,1);
						\node[blue] at (1,28) {$(1,28)$};
						\node[blue] at (-1,0) {$(-1,0)$};
						\node[blue] at (5,0) {$(5,0)$};							
					\end{axis}						
			\end{tikzpicture}
		\end{minipage}
	\end{figure}
	\begin{figure}[h]
		\begin{minipage}{0.5\textwidth}
			\subsubsection*{b)}
			\begin{align*}
				h(x) &= -0^{2} -0 * 2 + 27 \\
			\end{align*}
			\text{27 Meter}
			\subsubsection*{c)}
			\begin{align*}
				S(1 | 28 ) \\
			\end{align*}
			\hspace{25mm}
			\text{28 Meter}
		\end{minipage}
		\begin{minipage}{0.5\textwidth}
			\subsubsection*{d)}
			\begin{align*}
				h(x) &= x^{2} + 2x + 27 \\
				&= x^{2} - 2x -27 \\
				x_{1/2} &= -\frac{2}{2} \pm\sqrt{(\frac{-2}{2})^{2}+27} \\
				x_{1} &= 6,291 \\
				x_{2} &= -4,292 \\
			\end{align*}
			\text{Sie springt 6,292 Meter weit}
		\end{minipage}
	\end{figure}
\end{document}